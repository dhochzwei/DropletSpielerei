\documentclass[11pt,a4paper]{article}
\begin{document}
\subsection*{Infos}
\begin{itemize}
	\item Ground ist immer das dunklere Kabel.
	\item Am Raspberry Pi Passwort: \emph{keines}
	\item Teensy über Teensyduino flashen. Sicher gehen, dass Pure Data gerade nicht auf den USB Port zugreift.
	\item \emph{.pd}-Files sind die Pure Data Files.
	\item cheetomoskeeto -> Leert Pure Data und OSC.
	\item In Pure Data gibts es zwei Modes: 1. Cursor ist Hand: man kann Werte ändern. 2. Cursor ist Pfeil: man kann Knöpfchen drücken.
	\item WICHTIG: in Pure Data: Rechtsklick auf ein Objekt -> Hilfe. Gibt hilfreiche Infos
	\item Z-Koordinaten sind negativ: 0 ist ganz weit oben, -60000 ist ca im Fokus.
\end{itemize}

\subsection*{Startup}
\begin{enumerate}
	\item Power Supply beim Rig anschließen. LED leuchtet. Kamera ist bereit
	\item Im File \emph{software\_commands} stehen zwei Programme um die Kamera zu starten.
	\item Pure Data übers Terminal mit \emph{sudo} starten.
	\item Fenster maximieren um den Graphikbug zu beseitigen.
	\item Links oben auf Devices klicken: In der Console werden die Devices angezeigt. 
	\item unbedingt einmal links- rechts fahren, bevor man auf und ab fährt.
	\item Zum Schluss vor dem Ausschalten wieder auf 0 gehen ist praktisch.
\end{enumerate}
\subsection*{OpenCV}
\end{document}
